
\section{Installation}

    ks.py is a kernel search implementation written in Python, using Gurobi as back end MIP solver. In order to use it, you must
    install both Python 3.8, or above, and  Gurobi 9 and some other dependencies. Gurobi is the only commercial software used by ks.py. 
    ks.py can be used on any system supported by gurobipy (Gurobi's python bindings).
    
    \subsection{Install Python}
        The Python interpreter can be freely downloaded from the \href{https://www.python.org/}{official website}. It is mandatory to install 
        the 64 bit version, otherwise Gurobi cannot be linked to the interpreter. If it is possible to install Python 3.8 using your system's
        package manager, feel free to install with this method. Ensure to also install Pip with Python. Pip is the official Python's package manager,
        the third party library installer, it is needed to install ks.py dependencies.

        ks.py can be downloaded from \href{https://github.com/FilippoRanza/ks.py}{the official repository}. My advice is to download the most stable 
        version that fit your needs. So first consider to download the last stable \href{https://github.com/FilippoRanza/ks.py/releases}{tag}, if this is not enough download the last commit on 
        \href{https://github.com/FilippoRanza/ks.py/archive/master.zip}{master}. Avoid 
        the other branches especially the failing ones, unless you are considering to contribute to the project. It is possible to download ks.py anywhere 
        in your file system, but consider the it can be run (without issues) only from it's root directory.

        Once you've downloaded ks.py and place it into a proper location it's time to install ks.py dependencies. This is done using pip. 
        Open your system shell, Powershell on Windows, Terminal on Unix. Change directory \footnote{Usually in the file manager there is the possibility 
        to \emph{open a terminal here}. Otherwise in the file manager copy the path, open a shell, write \textbf{cd} a space and paste the path} where you've downloaded ks.py, there run this 
        command. 
        \begin{lstlisting}
            pip install -r requirements.txt
        \end{lstlisting}
        This may take a while. Be patient. This command can be run as standard or super user. In the first case this libraries will be installed locally to the 
        current user, in the second globally. This is up to you.

        \subsubsection{Troubleshooting}
            This is a good point to test your python installation.  A quick way to do this consist in opening a shell and run the command \textbf{python} (on Windows it may be \textbf{py}),
            at this point you should carefully check the banner printed by the python interpreter. There you must check that the version is 3.8 o above and that you are 
            using a 64 bit installation. Once done you can safely close the window. 

            It is possible for this procedure to fail or not be 100\% smooth. Some common issues are:
            \begin{enumerate}
                \item Pip is not installed: depending on your system and on your chosen installation method it is possible that pip is not installed as a
                separate command. In this scenario usually pip is installed as a python's sub module. So the command above becomes
               \begin{lstlisting}
    python -m pip install -r requirements.txt
                \end{lstlisting}
                
                \item On Windows the command \textbf{python} results in a \emph{command not found}: some installers on Windows install python as \textbf{py} 
                \item pip module not found: some installers are more minimalist then others, so it is possible that pip is not installed with python by default
                try to check again your installer's guide to see how to install pip.
            \end{enumerate}

    \subsection{Install Gurobi}
        Before installing Gurobi you must obtain a license, check on \href{https://www.gurobi.com/}{the official website}. Once you've obtained a license you can proceed 
        installing, and activating, Gurobi for your platform. Once done, it comes the trickiest step in this configuration: installing gurobipy, the official python bindings 
        for gurobi. This step can be done in to ways:
            \begin{itemize}
                \item The minimalist: using your system's cli shell change directory to Gurobi installation root\footnote{\href{https://packages.gurobi.com/9.0/README.txt}{Check here}}. There search for the file \emph{setup.py}. Once found run this command
                \begin{lstlisting}
    python setup.py install
                \end{lstlisting}
                once this is done, it is pretty fast, gurobipy is installed and ready.
                \item The easier: install the anaconda package manager, a third party python package manager, 
                and follow the \href{https://www.gurobi.com/documentation/9.0/quickstart_mac/ins_the_anaconda_python_di.html}{instruction} 
                provided by Gurobi.
            \end{itemize}



            \subsubsection{Troubleshooting}
                This is a good moment for testing gurobipy. A quick way to test it is to open a python interactive shell and import gurobipy. Open a python shell as before and 
                insert:
                \begin{lstlisting}
    import gurobipy
                \end{lstlisting}
                At this point if there are no message are shown you've correctly installed python and gurobipy. If the interpreter shows a message saying \emph{ModuleNotFoundError}
                this means that the gurobipy is not installed. In both cases you can safely close the window.\\ Some common issues are:
                \begin{enumerate}
                    \item wrong python version: especially on Unix is very common to have python already installed, but, except for some Linux distros, this default installation is not Python3.8,
                    in this scenario is possible that gurobipy is been installed for another python version. Possible fixes:
                    \begin{enumerate}
                        \item Change default python version to 3.8
                        \item Explicitly use python3.8 in lieu of python 
                        \item If possible remove old python installation (do this only is you are 110\% sure)
                    \end{enumerate}
                    \item wrong python compilation: for Windows python is compiled for both 32 and 64 bit. It is a common mistake to install python 32 bit
                    \item gurobipy is installed but is does not work: check that Gurobi is correctly activated using the \textbf{grbgetkey} command\
                \end{enumerate}
