\documentclass[12pt,a4paper]{article}

\usepackage{emptypage}
\usepackage[utf8]{inputenc}
\usepackage{amsmath}
\usepackage{dirtytalk}
\usepackage{listings}
\usepackage{tablefootnote}


\lstset{
frame = single, 
language=bash, 
framexleftmargin=-15pt,
framexrightmargin=-15pt}


\usepackage{hyperref}
\hypersetup{
    colorlinks=true,
    linkcolor=black,
    filecolor=magenta,      
    urlcolor=cyan,
}


\title{ks.py\\
\large User Guide}

\author{Filippo Ranza}
\date{}


\begin{document}

    \maketitle
    \begin{abstract}
        This document provides a complete user guide to ks.py. It 
        is organized as follows: the first section contains general installation 
        instruction, explaining in detail how to configure any dependency; the second
        section explains how to use ks.py; the third section shows how to properly understand
        ks.py and any given result; the last section contains some usage examples.
    \end{abstract}

    \tableofcontents
 
    
\section{Installation}

    ks.py is a kernel search implementation written in Python, using Gurobi as back end MIP solver. In order to use it, you must
    install both Python 3.8, or above, and  Gurobi 9 and some other dependencies. Gurobi is the only commercial software used by ks.py. 
    ks.py can be used on any system supported by gurobipy (Gurobi's python bindings).
    
    \subsection{Install Python}
        The Python interpreter can be freely downloaded from the \href{https://www.python.org/}{official website}. It is mandatory to install 
        the 64 bit version, otherwise Gurobi cannot be linked to the interpreter. If it is possible to install Python 3.8 using your system's
        package manager, feel free to install with this method. Ensure to also install Pip with Python. Pip is the official Python's package manager,
        the third party library installer, it is needed to install ks.py dependencies.

        ks.py can be downloaded from \href{https://github.com/FilippoRanza/ks.py}{the official repository}. My advice is to download the most stable 
        version that fit your needs. So first consider to download the last stable \href{https://github.com/FilippoRanza/ks.py/releases}{tag}, if this is not enough download the last commit on 
        \href{https://github.com/FilippoRanza/ks.py/archive/master.zip}{master}. Avoid 
        the other branches especially the failing ones, unless you are considering to contribute to the project. It is possible to download ks.py anywhere 
        in your file system, but consider the it can be run (without issues) only from it's root directory.

        Once you've downloaded ks.py and place it into a proper location it's time to install ks.py dependencies. This is done using pip. 
        Open your system shell, Powershell on Windows, Terminal on Unix. Change directory \footnote{Usually in the file manager there is the possibility 
        to \emph{open a terminal here}. Otherwise in the file manager copy the path, open a shell, write \textbf{cd} a space and paste the path} where you've downloaded ks.py, there run this 
        command. 
        \begin{lstlisting}
            pip install -r requirements.txt
        \end{lstlisting}
        This may take a while. Be patient. This command can be run as standard or super user. In the first case this libraries will be installed locally to the 
        current user, in the second globally. This is up to you.

        \subsubsection{Troubleshooting}
            This is a good point to test your python installation.  A quick way to do this consist in opening a shell and run the command \textbf{python} (on Windows it may be \textbf{py}),
            at this point you should carefully check the banner printed by the python interpreter. There you must check that the version is 3.8 o above and that you are 
            using a 64 bit installation. Once done you can safely close the window. 

            It is possible for this procedure to fail or not be 100\% smooth. Some common issues are:
            \begin{enumerate}
                \item Pip is not installed: depending on your system and on your chosen installation method it is possible that pip is not installed as a
                separate command. In this scenario usually pip is installed as a python's sub module. So the command above becomes
               \begin{lstlisting}
    python -m pip install -r requirements.txt
                \end{lstlisting}
                
                \item On Windows the command \textbf{python} results in a \emph{command not found}: some installers on Windows install python as \textbf{py} 
                \item pip module not found: some installers are more minimalist then others, so it is possible that pip is not installed with python by default
                try to check again your installer's guide to see how to install pip.
            \end{enumerate}

    \subsection{Install Gurobi}
        Before installing Gurobi you must obtain a license, check on \href{https://www.gurobi.com/}{the official website}. Once you've obtained a license you can proceed 
        installing, and activating, Gurobi for your platform. Once done, it comes the trickiest step in this configuration: installing gurobipy, the official python bindings 
        for gurobi. This step can be done in to ways:
            \begin{itemize}
                \item The minimalist: using your system's cli shell change directory to Gurobi installation root\footnote{\href{https://packages.gurobi.com/9.0/README.txt}{Check here}}. There search for the file \emph{setup.py}. Once found run this command
                \begin{lstlisting}
    python setup.py install
                \end{lstlisting}
                once this is done, it is pretty fast, gurobipy is installed and ready.
                \item The easier: install the anaconda package manager, a third party python package manager, 
                and follow the \href{https://www.gurobi.com/documentation/9.0/quickstart_mac/ins_the_anaconda_python_di.html}{instruction} 
                provided by Gurobi.
            \end{itemize}



            \subsubsection{Troubleshooting}
                This is a good moment for testing gurobipy. A quick way to test it is to open a python interactive shell and import gurobipy. Open a python shell as before and 
                insert:
                \begin{lstlisting}
    import gurobipy
                \end{lstlisting}
                At this point if there are no message are shown you've correctly installed python and gurobipy. If the interpreter shows a message saying \emph{ModuleNotFoundError}
                this means that the gurobipy is not installed. In both cases you can safely close the window.\\ Some common issues are:
                \begin{enumerate}
                    \item wrong python version: especially on Unix is very common to have python already installed, but, except for some Linux distros, this default installation is not Python3.8,
                    in this scenario is possible that gurobipy is been installed for another python version. Possible fixes:
                    \begin{enumerate}
                        \item Change default python version to 3.8
                        \item Explicitly use python3.8 in lieu of python 
                        \item If possible remove old python installation (do this only is you are 110\% sure)
                    \end{enumerate}
                    \item wrong python compilation: for Windows python is compiled for both 32 and 64 bit. It is a common mistake to install python 32 bit
                    \item gurobipy is installed but is does not work: check that Gurobi is correctly activated using the \textbf{grbgetkey} command\
                \end{enumerate}

    \section{Usage}

    ks.py is straightforward to use. ks.py provides a CLI interface, so it can be run from the shell 
    without the need of a dedicated development environment. ks.py is configured using a YAML
    \footnote{If you've never eared them take a look at the \href{https://en.wikipedia.org/wiki/YAML}{Wikipedia} page to learn the basic syntax, 
    this will avoid a lot of issues in a second moment } file,
    so it is easy to modify this configuration using any text editor. ks.py should be configured using its configuration file and not by modifying 
    its source code. 

    \subsection{Running}

        Run ks.py is equivalent on any platform. Open a shell, go to ks.py root directory and using \textbf{python} (or \textbf{py} on some Windows installation) run:
        \begin{lstlisting}
            python ks.py -h
        \end{lstlisting}
        This command will print on the screen the help message and immediately exit. As you can see ks.py expects a mandatory file, the instance file (usually .mps) and 
        an optional argument introduced with the flag \textbf{-c}, this is the configuration file (must be a YAML file). ks.py does not requires some names or extensions the important
        thing is that those files are correctly formatted. If you omit the configuration file ks.py uses its default configuration, in order to solve the instance \emph{problem.mps} use 
        the command:
        \begin{lstlisting}
            python ks.py problem.mps
        \end{lstlisting}
        ks.py default configuration may vary inadvertently with the version so you are highly discouraged from using it.

        In order to specify a configuration use this command:
        \begin{lstlisting}
            python ks.py -c config.yml problem.mps
        \end{lstlisting}
        Supposing that your configuration is in a file named \emph{config.yml} and your problem is in a file named \emph{problem.mps}. Remember instance and configuration file names are irrelevant
        to ks.py as the order you provide them.

        It is very likely that your configuration is unfortunate and is unable to provide solution in an acceptable amount of time. In this case you can arrest ks.py sending an interrupt signal 
        (on most platforms is \texttt{CTRL+C}), this will hardly kill the process so any progress is lost. 

        \subsubsection{Troubleshooting}
            Here I'll consider that the installation is correct, so any issue described in this section is due to error that may happen during the execution.
            Some common issues are:
            \begin{enumerate}
                \item Configuration fails to load: it is due to syntax error in the configuration file or to mismatched parameter type. Check the your configuration file
                follows the YAML syntax and that any value is correctly set
                \item Instance fail to load: this is a rare event, usually caused by a malformed mps (or analogous) file. Read the log error for more information.
                \item File not found: if configuration or instance file is not in the current working directory ks.py is not able to magically locate it. If you keep your 
                configuration or instances into an other directory you must use a fully qualified path. 
            \end{enumerate}


    \subsection{Configuration}
        To obtain some valuable results from ks.py, so a good objective value in a short amount of time, it is important to define a proper configuration file. 
        The configuration file is made of key-value pairs. 
        ks.py contains an example of configuration file. You can use it or keep it as a guide. In the rest of this section any configuration parameter will be 
        explained in detail. Some parameter are mandatory in certain contexts, other are always optional. The following table contains the list of currently 
        available commands. It is just a simple recap to remember any attribute type.
  
        \newpage

        \begin{table}[h]
            \centering
            \caption{Configuration parameters}
            \begin{tabular}{ | l | l | l | }
                \hline
                Attribute & Sub attribute & Type \\
                \hline
                \hline
                PRELOAD& & boolean\\ 
                \hline
                 LOG& & boolean\\ 
                \hline
                 BUCKET& & string\\ 
                \hline
                BUCKET\_CONF & & structured\\ 
                \hline
                & count & integer\\ 
                \hline
                & size & integer\\ 
                \hline
                 ITERATIONS& & integer\\ 
                \hline
                 DEBUG& & string\\ 
                \hline
                 SOLUTION\_FILE& & string\\ 
                \hline
                 MIP\_GAP& & float\\ 
                \hline
                FEATURE\_KERNEL & & structured\\ 
                \hline
                & COUNT & integer\\ 
                \hline
                & MIN\_TIME & integer\\ 
                \hline
                & MAX\_TIME & integer\\ 
                \hline
                & POLICY & string\\ 
                \hline
                & LOG\_FILE & string\\ 
                \hline
                & CACHE\_FILE & string\\ 
                \hline        
            \end{tabular}
            \label{tab:params}
        \end{table}
    
        \subsubsection{Parameter Details}
    
        \begin{itemize}
            \item PRELOAD: specify to ks.py if preset the current solution using the previous solution (from  previous bucket, initial kernel or from file).
            \item LOG: specify to ks.py to enable or disable Gurobi log, if \textbf{on} the execution is very verbose and rich of potentially useful information.
            \item BUCKET: specify to ks.py how to build the buckets.
            \item BUCKET\_CONF : specify the bucket builder configuration
            \begin{itemize}
               \item count: specify to the bucket builder the number of buckets. If size is set this parameter should not be set
               \item size: specify bucket size. If count is set this parameter should not be set 
            \end{itemize}
            \item ITERATIONS: specify to ks.py how many bucket iteration to perform
            \item DEBUG: specify ks.py the optional debug file. It is a CSV file containing the convergence story from bucket iteration
            \item SOLUTION\_FILE: specify ks.py the optional solution file. It is a Gurobi SOL file, a simple text file containing on each line
            a couple \emph{variable name} - \emph{value}, useful also for successive execution.
            \item MIP\_GAP: specify the minimal mip gap, as a percentage value, when it goes under this level the solution is considered optimal. The current \emph{mip\_gap}
            is computed as:
            \begin{align*}
                &\frac{UB - LB}{|UB|}\\
                Where:&\\
                &UB \text{ is the current upper objective bound}\\
                &LB \text{ is the current lower objective bound}\\
            \end{align*} 
            So the current sub problem is considered solved at optimal if
            \begin{align*}
                &\frac{UB - LB}{|UB|} < MIP\_GAP
            \end{align*} 
            and terminate the execution.

            \item FEATURE\_KERNEL : use feature kernel to build the initial kernel instead of the kernel build from base variables in the LP relaxation.
            \begin{itemize}
               \item COUNT: specify the number of random sub problem to run and use to compute the initial kernel ,
               \item MIN\_TIME: specify the starting time, in seconds, to solve a random sub problem. If MAX\_TIME is set also MIN\_TIME must be set
               \item MAX\_TIME: specify the maximum time, in seconds, to solve a random sub problem. If MIN\_TIME is set also MAX\_TIME must be set
               \item POLICY: specify the policy used to find the size of the initial kernel.
               \item LOG\_FILE: specify the optional file name of the feature kernel log file, it is a CSV file containing the status of each sub problem.
               \item CACHE\_FILE: specify the optional file name of the sub problem cache file. This file contains any sub problem, with variable values and 
               optimization status.
            \end{itemize}
         \end{itemize}


        
        \subsubsection{Parameter Details}
        Some parameters requires a more rich description. 
        \paragraph{BUCKET} specify the method used to build the bucket. The available method are described in Table \ref{tab:bucket}

        \begin{table}[ht]
            \centering
           \caption{BUCKET Values}
           \begin{tabular}{|l|l|}
               \hline 
               Value & Description \\
               \hline
               \hline
               fixed & Create a set of fixed size buckets\\
               \hline
               decrease & Create a set of decreasing size buckets\\
               &  each bucket size double the size of its successor\\
               \hline
           \end{tabular}
           \label{tab:bucket}
        \end{table}

        Each method supports the same \textbf{BUCKET\_CONF}. In any case if you set \emph{count} ks.py will create \emph{count} buckets the size 
        of each bucket is automatically calculated according to specified method and the number of variables outside kernel.
        If you specify \emph{size} the behavior with \emph{fixed} is obvious, ks.py will create a number of buckets each with \emph{size} variables but
        with \emph{decrease} it is less obvious: in this case ks.py uses \emph{size} as the size of the smallest bucket (the last one), then according to
        this size it will compute the correct number of bucket. Any eventually remainder is put into the last bucket. 

        It is possible that the configuration (especially when using \emph{size}) requires more buckets then the bucket outside the kernel. In this case 
        ks.py detects the error and immediately stops the execution. This is considered as an error so the initial kernel is not kept, unless using Feature Kernel
        with \textbf{CACHE\_FILE} set.



        \paragraph{Output Files} ks.py can optionally output four file. Each file has a fixed format and cannot be changed by the configuration file. Any file 
        name will be blindly accepted by ks.py, so some point must be understood.
        \begin{itemize}
            \item Running ks.py without changing file names will cause ks.py to overwrite the previous contents, leading into an unrepeatable lost. Is up to 
            you to properly reconfigure, rename or store the files.
            \item File extension is meaningless to ks.py, you should pay attention to the extensions so the system can handle the files properly.
        \end{itemize} 


        Any output file is in a public domain format and can be opened with many software, both open source and proprietary. The only non textual file 
        is the \textbf{CACHE\_FILE} which is a \href{https://docs.python.org/3/library/pickle.html}{pickle} file, the default serialization library in Python.
        \href{https://en.wikipedia.org/wiki/Comma-separated_values}{CSV} are used because this format is very easy to parse with almost any  programming language, 
        and most of the languages already have a library to parse them. It is also possible to open CSV using tools like \href{https://docs.google.com/spreadsheets}{Google Sheets}. 
      
        \begin{table}[ht]
            \centering
            \caption{File type description}
            \begin{tabular}{|l|l|}
                \hline
                Parameter & Type \\
                \hline 
                DEBUG & CVS - Textual\\
                \hline 
                SOLUTION\_FILE & SOL - Textual\\
                \hline 
                LOG\_FILE & CSV - Textual \\
                \hline 
                CACHE\_FILE & Pickle - Binary \\
                \hline
            \end{tabular}
            \label{tab:file_decr1}
        \end{table}

        How to properly analyze and understand the contents of this file, especially DEBUG and LOG\_FILE, will be explained in the following session.


        \subsubsection{Troubleshooting}
        Although YAML syntax does not check for types, ks.py does. A common problem that may arise is a mistypes parameter value, infarct to ks.py $4$ and $4.0$
        is not the same value nor the same type. So where Table \ref{tab:params} specifies a \emph{float} always add a decimal part (eventually $.0$), when a \emph{string} 
        is required always add marks and avoid decimal part for \emph{integers}.

        Other configuration may result into an undesired behavior, like been stuck into some slowly converging solution. This is not a real \emph{error} but more 
        an unfortunate condition that must be addressed by changing the configuration.

    \section{Understanding Results}
    ks.py optionally produces five different outputs. They are:
    \begin{enumerate}
        \item Bucket Convergence History
        \item Feature Kernel Sub problem Status
        \item Solution File
        \item Feature Kernel Cache File 
        \item Console Log
    \end{enumerate}
    
    \subsubsection{Bucket Convergence History} 
        This output logs the convergence history of the bucket iteration. It is configured using the parameter 
        \textbf{DEBUG}, see Table \ref{tab:params}. When this parameter is enabled will log any successful bucket iteration into a CSV file containing the following
        values:
            \begin{table}[h]
                \centering
                \caption{Debug File Parameters}
                \begin{tabular}{|l|l|l|}
                    \hline
                    Value & Description & Type \\
                    \hline
                    \hline
                    bucket & current bucket id  & Integer\\
                    \hline
                    iteration & current iteration  & Integer\\
                    \hline
                    value & objective function &  Variable\footnotemark\\
                    \hline
                    time & required time, in seconds & Integer \\
                    \hline
                    nodes & number of explored nodes & Integer\\
                    \hline
                    kernel\_size & number of variable in the kernel & Integer\\
                    \hline
                    bucket\_size & number of variable in the current bucket & Integer \\
                    \hline
                \end{tabular}
            \end{table} 
            \footnotetext{Depends on the specific instance}
        
        This file contains various useful clues that will help to understand if the current configuration is good or not. For example is possible to view how the 
        objective value varies, with the buckets and the time. This should help to address some issues in the bucket structure. Also the number of explored nodes, especially
        if correlated with the required time can show if the configuration is fine or not. Those data can be collected with data from other executions
        to understand if the issue in finding a good solution comes from a bad configuration or if the Kernel Search itself is not the right method to solve it.

        Some issues that can be found, and addressed using this data are:
        \begin{itemize}
            \item Bucket to small: in this case the objective value does not change by a relevant amount for a lot of iterations.
            \item Bucket to large: in this case the required time for each single problem is to large and the number of explored nodes is very high
            \item Mip Gap to small: long execution time and small improvements in the objective function can also mean that the Mip gap is to small. Try to increase it and let
            be Kernel Search to improve this value not Gurobi
            \item Too many iterations: it is possible that after a certain point the value of the objective function is stuck to a certain value, this means the ks.py reached a
            local minima and increasing the number of iteration cannot lead to a better solution. 
            \item Bucket method is wrong: it is possible that many sub problems are infeasible (they are not shown into the list), in this case is possible that a different 
            bucket or kernel builder can produce better results
        \end{itemize}


        This are only some of the possible usage of this output data. Depending on the specific instance and values of this data many other information can be gained, especially
        when considering also other outputs.

    \subsubsection{Feature Kernel Sub problem Status}
        This output log the final status of each random sub problem. It is configured using the parameter \textbf{LOG\_FILE}, see Table \ref{tab:params}. When this output
        is enabled it will produce a CSV file made of this fields:
        \begin{table}[h]
            \centering
            \caption{Feature Kernel Log Fields}
            \begin{tabular}{|l|l|l|}
                \hline
                Field & Description & Type \\
                \hline
                \hline
                Iteration & Iteration Index & Integer \\
                \hline
                Problem Size & Variables in current sub problem & Integer \\
                \hline
                Status & Status of the sub problem & Integer (see \ref{tab:fk-status}) \\
                \hline
            \end{tabular}
        \end{table}

        
        \begin{table}[h]
            \centering
            \caption{Feature Kernel Status}
            \begin{tabular}{|l|l|}
                \hline
                Value & Description \\
                \hline
                \hline
                & problem infeasible \\
                \hline
                0 & relaxation feasible, integer infeasible \\
                \hline
                1 & integer feasible \\
                \hline
                2 & time out \\
                \hline 
            \end{tabular}
            \label{tab:fk-status}
        \end{table}


        Table \ref{tab:fk-status} show the meaning of the various final statuses. This information combined with the size of the 
        sub problem can provide big hints into the problem. It is possible to understand an approximated threshold of both continuous 
        and integer feasibility. This value is partially influenced by the timeout values and by the randomness of the method, but if 
        the number of random sub problems is high enough conclusions from this method can be considered quite correct. 

        It is possible to define the Continuous Feasibility Threshold (CFT) as the average value of the size of sub problems feasible in the continuous but 
        not in the integer domain divided by the size of the whole model. Analogously it is possible to define Integer Feasibility Threshold (IFT)
        as the average value of the size of sub problems feasible in the integer domain but domain divided by the size of the whole model. Those two values
        can provide a good idea about the problem. For example a very high CFT and IFT means that the sub problems hardly lead to feasible solution, so only a 
        sub problem very close to the whole problem can be feasible. In this case probably Kernel Search is not the right method and is very unlikely that
        changing some parameters can lead to improvements. On the other side if CFT and IFT are small that means that for this particular problem Kernel Search 
        can provide a good method to find in a quick time a good solution. 

        It is possible that none of the sub problems results to be continuous or integer feasible (sometimes both) in this case the relative index is simply 
        undefined. 

        You should also check for \emph{time out} statuses. This means that the given time is not enough and that should be increased. You should start by setting 
        \emph{min} and \emph{max} to some standard values like 10 seconds and 30 seconds then start adjusting those values accordingly to the final status. 
        The general rule of the thumb is to increase the \emph{max} time if many problems and with \emph{timeout}, then when the difference between \emph{max} and \emph{min}
        becomes larger then the number of iteration it is possible to increase \emph{min} time. If you start going to high 
        probably Feature Kernel is not the right method.
        
        The initial count should be set to a value that is not too large nor too small, like 50. Once you start getting some useful results, like a uniform enough distribution
        of the various statuses, you can consider to increase this number in order to create a large pool of sub problems and create a meaningful initial kernel. 


    \subsubsection{Solution File}
        This output generate a \href{https://en.wikipedia.org/wiki/Sol_(format)}{SOL} file, an open format used to describe the solution of mathematical programming
        problem. This file is automatically used by ks.py to load a previous solution on the first iteration of successive run, if present. This file can also be 
        used to export and use the final solution. 

    \subsubsection{Feature Kernel Cache File}
        This output generate a cache file for the Feature Kernel sub problems. In some cases FK is may take a very long time. This is not an issue by itself, but 
        when the bucket configuration happens to be wrong (this happens more often then not) recomputing the some initial kernel over and over becomes annoying and
        time wasteful. To address this issue you can set the cache file and once the initial kernel is satisfying you can (without removing the cache file or touching
        this parameter) set \textbf{COUNT} to zero. This will create zero sub problem and simply load the previously found and then build the initial kernel from there. 
        Because those sub problems are randomly generated a configuration cannot be better than the other so
        the cache file accumulates any generated model, so if you run FK six times with \textbf{COUNT} set to fifty
        the cache file will contain three hundred sub problems. This trick allows to create the initial kernel once, without the need to recompute it for each different 
        bucket configuration. 


    \subsubsection{Console Log}
        This output is mainly composed of the Gurobi logs, and some ks.py error messages. At the current time those data can only be accessed as screen log and 
        the only way to store them all is by piping the stdout into a file. This must be done by the parent program. 

        The Gurobi log contains some of the most useful information. In fact this log contains a detailed convergence history for each Sub problem, both during 
        the initial kernel generation and during the bucket iterations. A number of very useful information can be obtained looking at this logs:
        \begin{itemize}
            \item The MIP Gap is too large: the problem reaches a good solution quickly but then it remains stuck and requires a long time to improve the solution
            \item The Buckets are misconfigured: this happens when:
            \begin{itemize}
                \item it requires a long time to find that this sub problem is infeasible
                \item it takes a long time to solve a sub problem
                \item the solution slowly converges
            \end{itemize}
            \item Time Limit is to small: some problem cannot be solved because the time limit is too small for this specific instance 
        \end{itemize}

    
    \subsubsection{Final Guidelines on Outputs}
        Except for \emph{solution} and \emph{cache} file that are used to export a result, the other outputs are useful only to guide you to find a 
        better configuration and to understand if the instance is suitable for Kernel Search or not. In general it is a good idea to keep any output 
        enabled and take a look at all of them to better understand how to proceed. 
        My final advice is to start solving some \emph{easy} benchmark instance before solving your real problems, this will give you a bit of experience 
        in understanding the results from ks.py. 













    






        


    



\end{document}