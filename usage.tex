\section{Usage}

    ks.py is straightforward to use. ks.py provides a CLI interface, so it can be run from the shell 
    without the need of a dedicated development environment. ks.py is configured using a YAML
    \footnote{If you've never eared them take a look at the \href{https://en.wikipedia.org/wiki/YAML}{Wikipedia} page to learn the basic syntax, 
    this will avoid a lot of issues in a second moment } file,
    so it is easy to modify this configuration using any text editor. ks.py should be configured using its configuration file and not by modifying 
    its source code. 

    \subsection{Running}

        Run ks.py is equivalent on any platform. Open a shell, go to ks.py root directory and using \textbf{python} (or \textbf{py} on some Windows installation) run:
        \begin{lstlisting}
            python ks.py -h
        \end{lstlisting}
        This command will print on the screen the help message and immediately exit. As you can see ks.py expects a mandatory file, the instance file (usually .mps) and 
        an optional argument introduced with the flag \textbf{-c}, this is the configuration file (must be a YAML file). ks.py does not requires some names or extensions the important
        thing is that those files are correctly formatted. If you omit the configuration file ks.py uses its default configuration, in order to solve the instance \emph{problem.mps} use 
        the command:
        \begin{lstlisting}
            python ks.py problem.mps
        \end{lstlisting}
        ks.py default configuration may vary inadvertently with the version so you are highly discouraged from using it.

        In order to specify a configuration use this command:
        \begin{lstlisting}
            python ks.py -c config.yml problem.mps
        \end{lstlisting}
        Supposing that your configuration is in a file named \emph{config.yml} and your problem is in a file named \emph{problem.mps}. Remember instance and configuration file names are irrelevant
        to ks.py as the order you provide them.

        It is very likely that your configuration is unfortunate and is unable to provide solution in an acceptable amount of time. In this case you can arrest ks.py sending an interrupt signal 
        (on most platforms is \texttt{CTRL+C}), this will hardly kill the process so any progress is lost. 

        \subsubsection{Troubleshooting}
            Here I'll consider that the installation is correct, so any issue described in this section is due to error that may happen during the execution.
            Some common issues are:
            \begin{enumerate}
                \item Configuration fails to load: it is due to syntax error in the configuration file or to mismatched parameter type. Check the your configuration file
                follows the YAML syntax and that any value is correctly set
                \item Instance fail to load: this is a rare event, usually caused by a malformed mps (or analogous) file. Read the log error for more information.
                \item File not found: if configuration or instance file is not in the current working directory ks.py is not able to magically locate it. If you keep your 
                configuration or instances into an other directory you must use a fully qualified path. 
            \end{enumerate}


    \subsection{Configuration}
        To obtain some valuable results from ks.py, so a good objective value in a short amount of time, it is important to define a proper configuration file. 
        The configuration file is made of key-value pairs. 
        ks.py contains an example of configuration file. You can use it or keep it as a guide. In the rest of this section any configuration parameter will be 
        explained in detail. Some parameter are mandatory in certain contexts, other are always optional. The following table contains the list of currently 
        available commands. It is just a simple recap to remember any attribute type.
  
        \newpage

        \begin{table}[h]
            \centering
            \caption{Configuration parameters}
            \begin{tabular}{ | l | l | l | }
                \hline
                Attribute & Sub attribute & Type \\
                \hline
                \hline
                PRELOAD& & boolean\\ 
                \hline
                 LOG& & boolean\\ 
                \hline
                 BUCKET& & string\\ 
                \hline
                BUCKET\_CONF & & structured\\ 
                \hline
                & count & integer\\ 
                \hline
                & size & integer\\ 
                \hline
                 ITERATIONS& & integer\\ 
                \hline
                 DEBUG& & string\\ 
                \hline
                 SOLUTION\_FILE& & string\\ 
                \hline
                 MIP\_GAP& & float\\ 
                \hline
                FEATURE\_KERNEL & & structured\\ 
                \hline
                & COUNT & integer\\ 
                \hline
                & MIN\_TIME & integer\\ 
                \hline
                & MAX\_TIME & integer\\ 
                \hline
                & POLICY & string\\ 
                \hline
                & LOG\_FILE & string\\ 
                \hline
                & CACHE\_FILE & string\\ 
                \hline        
            \end{tabular}
            \label{tab:params}
        \end{table}
    
        \subsubsection{Parameter Details}
    
        \begin{itemize}
            \item PRELOAD: specify to ks.py if preset the current solution using the previous solution (from  previous bucket, initial kernel or from file).
            \item LOG: specify to ks.py to enable or disable Gurobi log, if \textbf{on} the execution is very verbose and rich of potentially useful information.
            \item BUCKET: specify to ks.py how to build the buckets.
            \item BUCKET\_CONF : specify the bucket builder configuration
            \begin{itemize}
               \item count: specify to the bucket builder the number of buckets. If size is set this parameter should not be set
               \item size: specify bucket size. If count is set this parameter should not be set 
            \end{itemize}
            \item ITERATIONS: specify to ks.py how many bucket iteration to perform
            \item DEBUG: specify ks.py the optional debug file. It is a CSV file containing the convergence story from bucket iteration
            \item SOLUTION\_FILE: specify ks.py the optional solution file. It is a Gurobi SOL file, a simple text file containing on each line
            a couple \emph{variable name} - \emph{value}, useful also for successive execution.
            \item MIP\_GAP: specify the minimal mip gap, as a percentage value, when it goes under this level the solution is considered optimal. The current \emph{mip\_gap}
            is computed as:
            \begin{align*}
                &\frac{UB - LB}{|UB|}\\
                Where:&\\
                &UB \text{ is the current upper objective bound}\\
                &LB \text{ is the current lower objective bound}\\
            \end{align*} 
            So the current sub problem is considered solved at optimal if
            \begin{align*}
                &\frac{UB - LB}{|UB|} < MIP\_GAP
            \end{align*} 
            and terminate the execution.

            \item FEATURE\_KERNEL : use feature kernel to build the initial kernel instead of the kernel build from base variables in the LP relaxation.
            \begin{itemize}
               \item COUNT: specify the number of random sub problem to run and use to compute the initial kernel ,
               \item MIN\_TIME: specify the starting time, in seconds, to solve a random sub problem. If MAX\_TIME is set also MIN\_TIME must be set
               \item MAX\_TIME: specify the maximum time, in seconds, to solve a random sub problem. If MIN\_TIME is set also MAX\_TIME must be set
               \item POLICY: specify the policy used to find the size of the initial kernel.
               \item LOG\_FILE: specify the optional file name of the feature kernel log file, it is a CSV file containing the status of each sub problem.
               \item CACHE\_FILE: specify the optional file name of the sub problem cache file. This file contains any sub problem, with variable values and 
               optimization status.
            \end{itemize}
         \end{itemize}


        
        \subsubsection{Parameter Details}
        Some parameters requires a more rich description. 
        \paragraph{BUCKET} specify the method used to build the bucket. The available method are described in Table \ref{tab:bucket}

        \begin{table}[ht]
            \centering
           \caption{BUCKET Values}
           \begin{tabular}{|l|l|}
               \hline 
               Value & Description \\
               \hline
               \hline
               fixed & Create a set of fixed size buckets\\
               \hline
               decrease & Create a set of decreasing size buckets\\
               &  each bucket size double the size of its successor\\
               \hline
           \end{tabular}
           \label{tab:bucket}
        \end{table}

        Each method supports the same \textbf{BUCKET\_CONF}. In any case if you set \emph{count} ks.py will create \emph{count} buckets the size 
        of each bucket is automatically calculated according to specified method and the number of variables outside kernel.
        If you specify \emph{size} the behavior with \emph{fixed} is obvious, ks.py will create a number of buckets each with \emph{size} variables but
        with \emph{decrease} it is less obvious: in this case ks.py uses \emph{size} as the size of the smallest bucket (the last one), then according to
        this size it will compute the correct number of bucket. Any eventually remainder is put into the last bucket. 

        It is possible that the configuration (especially when using \emph{size}) requires more buckets then the bucket outside the kernel. In this case 
        ks.py detects the error and immediately stops the execution. This is considered as an error so the initial kernel is not kept, unless using Feature Kernel
        with \textbf{CACHE\_FILE} set.



        \paragraph{Output Files} ks.py can optionally output four file. Each file has a fixed format and cannot be changed by the configuration file. Any file 
        name will be blindly accepted by ks.py, so some point must be understood.
        \begin{itemize}
            \item Running ks.py without changing file names will cause ks.py to overwrite the previous contents, leading into an unrepeatable lost. Is up to 
            you to properly reconfigure, rename or store the files.
            \item File extension is meaningless to ks.py, you should pay attention to the extensions so the system can handle the files properly.
        \end{itemize} 


        Any output file is in a public domain format and can be opened with many software, both open source and proprietary. The only non textual file 
        is the \textbf{CACHE\_FILE} which is a \href{https://docs.python.org/3/library/pickle.html}{pickle} file, the default serialization library in Python.
        \href{https://en.wikipedia.org/wiki/Comma-separated_values}{CSV} are used because this format is very easy to parse with almost any  programming language, 
        and most of the languages already have a library to parse them. It is also possible to open CSV using tools like \href{https://docs.google.com/spreadsheets}{Google Sheets}. 
      
        \begin{table}[ht]
            \centering
            \caption{File type description}
            \begin{tabular}{|l|l|}
                \hline
                Parameter & Type \\
                \hline 
                DEBUG & CVS - Textual\\
                \hline 
                SOLUTION\_FILE & SOL - Textual\\
                \hline 
                LOG\_FILE & CSV - Textual \\
                \hline 
                CACHE\_FILE & Pickle - Binary \\
                \hline
            \end{tabular}
            \label{tab:file_decr1}
        \end{table}

        How to properly analyze and understand the contents of this file, especially DEBUG and LOG\_FILE, will be explained in the following session.


        \subsubsection{Troubleshooting}
        Although YAML syntax does not check for types, ks.py does. A common problem that may arise is a mistypes parameter value, infarct to ks.py $4$ and $4.0$
        is not the same value nor the same type. So where Table \ref{tab:params} specifies a \emph{float} always add a decimal part (eventually $.0$), when a \emph{string} 
        is required always add marks and avoid decimal part for \emph{integers}.

        Other configuration may result into an undesired behavior, like been stuck into some slowly converging solution. This is not a real \emph{error} but more 
        an unfortunate condition that must be addressed by changing the configuration.
